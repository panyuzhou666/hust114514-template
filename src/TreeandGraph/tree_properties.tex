% src/TreeandGraph/tree_properties.tex
% 包含树的重心和直径的核心结论
% 更新:移除了三级目录,增大了字体,并减小了行距

\begin{spacing}{0.95} % 减小行距 (主文件默认是 1.15, 可在 0.8-1.0 之间微调)
	\large % 将这部分内容的字体调大一级
	
	\subsection{树的重心}
	\paragraph{定义}
	\begin{itemize}
		\item \textbf{定义一 (按子树大小):} 找到一个点,删去这个点后剩下的所有子树的节点数都不超过总结点数的一半($\lfloor n/2 \rfloor$),那么这个点就是树的重心。
		\item \textbf{定义二 (按最大子树):} 找到一个点,其所有的子树中最大的子树节点数最少,那么这个点就是树的重心。
	\end{itemize}
	
	\paragraph{性质}
	\begin{itemize}
		\item \textbf{数量:} 一棵树最少有一个重心,最多只有两个重心。当且仅当存在一条边,断开后形成两个大小均为 $n/2$ 的子树时,这条边连接的两个节点都是重心。
		\item \textbf{距离和最小:} 树中所有点到重心的距离之和是最小的。如果有两个重心,那么它们的距离和相等。
		\item \textbf{路径性:} 把两棵树通过一条边相连得到一棵新的树,那么新树的重心在连接原来两棵树的重心的路径上。
		\item \textbf{动态变化:} 在一棵树上添加或删除一个叶子,它的重心最多只移动一条边的距离。
	\end{itemize}
	
	\subsection{树的直径}
	\paragraph{定义}
	树上的最长简单路径。(以下性质基于边权为正)
	
	\paragraph{性质}
	\begin{itemize}
		\item \textbf{中心性:} 如果一棵树有多条直径,那么它们必然相交于一点(这个点可能是某条边的中点)。
		\item \textbf{最远点:} 对于树上任意一点,距离其最远的点一定是某条直径的端点之一。
		\item \textbf{可合并性 (重要):} 对于树上的两个点集 $S$ 和 $T$,若 $S$ 中最远点对是 $(x,y)$,$T$ 中最远点对是 $(u,v)$,那么 $S \cup T$ 中的最远点对一定是 $\{x, y, u, v\}$ 这四个点中某两点组成的点对。
	\end{itemize}
	
\end{spacing}
