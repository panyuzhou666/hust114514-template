\paragraph{核心思想}
将树的拓扑结构映射成一个数。

\begin{itemize}
    \item 有根树比较好用的 hash 公式是 $f(p) = \sum_{s \in \text{son}(p)} \text{hash}(f(s)) + C$。
    \item 其中 \texttt{hash} 函数可以是随机数或者多项式。$C$ 是常数,取 1 就行。
    \item 这样的 hash 公式利于换根,换根公式也容易推导:
    \[
    g(p) = 
    \begin{cases} 
        f(p) & \text{p is root} \\ 
        f(p) + \text{hash}(g(\text{fa}(p)) - \text{hash}(f(p))) & \text{others} 
    \end{cases}
    \]
    \item 有根树的 hash 可以用根节点的 hash 值。
    \item 无根树的 hash 可以用所有有点作为根节点的 hash 值的异或和作为 hash 值。
\end{itemize}

\paragraph{复杂度分析}
\begin{itemize}[leftmargin=*]
    \item 随机数 hash 使用了 \texttt{map},复杂度为 $O(n \log n)$。
\end{itemize}

\paragraph{注意事项}
\begin{itemize}[leftmargin=*]
    \item 防止被卡,可以更改 \texttt{hash} 函数,常数 $C$,取模底数(如模 998244353),此时无根树 hash 可以改为所有点 hash 值的加和。
    \item 使用前记得调用 \texttt{calc(int root = 0)},否则会 \texttt{assert}。
\end{itemize}
