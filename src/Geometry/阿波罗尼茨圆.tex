\noindent 所有关于两点 $A,B$ 满足$PA/PB=k$且不等于$1$的点$P$的轨迹是一个圆.\\
圆幂: 半径为 $R$ 的圆 $O$, 任意一点 $P$ 到 $O$ 的幂为 $h=OP^2-R^2$\\
圆幂定理: 过 $P$ 的直线交圆在 $A$ 和 $B$ 两点, 则 $PA\cdot PB=|h|$\\
根轴: 到两圆等幂点的轨迹是一条垂直于连心线的直线\\
反演: 已知一圆 $C$, 圆心为 $O$, 半径为 $r$, 如果 $P$与 $P'$在过圆心 $O$的直线上, 且 $OP\cdot OP'=r^2$, 则称 $P$与 $P'$关于 $O$互为反演. 一般 $C$取单位圆.\\
反演的性质: \\
不过反演中心的直线反形是过反演中心的圆, 反之亦然.\\
不过反演中心的圆, 它的反形是一个不过反演中心的圆.\\
两条直线在交点 $A$的夹角, 等于它们的反形在相应点 $A'$的夹角, 但方向相反.\\
两个相交圆周在交点 $A$的夹角等于它们的反形在相应点 $A'$的夹角, 但方向相反.\\
直线和圆周在交点 $A$的夹角等于它们的反演图形在相应点 $A'$的夹角, 但方向相反.\\
正交圆反形也正交. 相切圆反形也相切, 当切点为反演中心时, 反形为两条平行线.
两两相切的圆 r1, r2, r3, 求与他们都相切的圆 r4.
分母取负号, 答案再取绝对值, 为外切圆半径.
分母取正号为内切圆半径.
$ r^{\pm}_4 = \frac{r_1 r_2 r_3}{r_1 r_2 + r_1 r_3 + r_2 r_3 \pm 2\sqrt{r_1r_2r_3(r_1 + r_2 + r_3)}} $
