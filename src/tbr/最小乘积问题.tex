\subsubsection{最小乘积问题原理}
\noindent
每个元素有两个权值$\{x_i\}$和$\{y_i\}$, 
要求在某个限制下(例如生成树, 二分图匹配)使得${\Sigma x \Sigma y}$最小.
对于任意一种符合限制的选取方法, 记$X=\Sigma x_i, Y=\Sigma y_i$, 可看做平面内一点$(X,Y)$.
答案必在下凸壳上, 找出该下凸壳所有点, 即可枚举获得最优答案.
可以递归求出此下凸壳所有点, 分别找出距 $x, y$ 轴最近的两点 $A, B$,
分别对应于$\Sigma y_i$, $\Sigma x_i$最小.
找出距离线段最远的点$C$, 则$C$也在下凸壳上, $C$点满足$AB\times AC$最小,
也即$$(X_B-X_A)Y_C + (Y_A-Y_B)X_C - (X_B-X_A)Y_A - (Y_B-Y_A)X_A$$最小,
后两项均为常数, 因此将所以权值改成$(X_B-X_A)y_i+(Y_B-Y_A)x_i$,求同样问题(例如最小生成树, 最小权匹配)即可.
求出$C$点以后, 递归$AC$, $BC$.
